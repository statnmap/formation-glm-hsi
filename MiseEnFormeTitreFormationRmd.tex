% ----------
% MISE EN FORME DES TITRES
% ----------
%\renewcommand{\thesection}{\arabic{section}}
%\renewcommand{\thesubsection}{\arabic{section}.\arabic{subsection}}
%\renewcommand{\thesubsubsection}{\arabic{section}.\arabic{subsection}.\arabic{subsubsection}}

%% Provide a definition to \subparagraph to keep titlesec happy
\let\subparagraph\oldsubparagraph
\let\paragraph\oldparagraph
%% Load titlesec
%\usepackage[compact]{titlesec}
\usepackage{titlesec}
%% Revert \subparagraph to the Rmd definition
% Redefines (sub)paragraphs to behave more like sections
\ifx\paragraph\undefined\else
\let\oldparagraph\paragraph
\renewcommand{\paragraph}[1]{\oldparagraph{#1}\mbox{}}
\fi
\ifx\subparagraph\undefined\else
\let\oldsubparagraph\subparagraph
\renewcommand{\subparagraph}[1]{\oldsubparagraph{#1}\mbox{}}
\fi


%\titleformat{(command)}[(shape)]{(format)}{(label)}{(sep)}{(before)}[(after)]
\titleformat{\section}%
[block]% style du titre (block, hang, display, runin, leftmargin, drop, wrap)
{\Large\color[RGB]{21,183,214}}%changement de fonte commun au numéro et au titre
{\thesection.}% spécification du numéro
{1em}% 1em espace entre le numéro et le titre
{}% changement de fonte du titre

\titleformat{\subsection}%
[block]% style du titre (block, hang, display, runin, leftmargin, drop, wrap)
{\large\bfseries} %\itshape {\Large\bfseries\color[RGB]{30,115,190}}%changement de fonte commun au numéro et au titre
{\thesubsection.}% spécification du numéro
{1em}% {1em}% espace entre le numéro et le titre
{}% changement de fonte du titre

\titleformat{\subsubsection}%
[block]% style du titre (block, hang, display, runin, leftmargin, drop, wrap)
{\large} %\itshape {\Large\bfseries\color[RGB]{30,115,190}}%changement de fonte commun au numéro et au titre
{\thesubsubsection.}% spécification du numéro
{1em}% {1em}% espace entre le numéro et le titre
{}% changement de fonte du titre

\titleformat{\paragraph}%
[block]% style du titre (block, hang, display, runin, leftmargin, drop, wrap)
{\itshape} %\itshape {\Large\bfseries\color[RGB]{30,115,190}}%changement de fonte commun au numéro et au titre
{\theparagraph.}% spécification du numéro
{1em}% {1em}% espace entre le numéro et le titre
{}% changement de fonte du titre

\titleformat{\subparagraph}%
[block]% style du titre (block, hang, display, runin, leftmargin, drop, wrap)
{\itshape} %\itshape {\Large\bfseries\color[RGB]{30,115,190}}%changement de fonte commun au numéro et au titre
{\thesubparagraph.}% spécification du numéro
{1em}% {1em}% espace entre le numéro et le titre
{}% changement de fonte du titre


%\titlespacing*{(command)}{(left)}{(beforesep)}{(aftersep)}[(right)]
\titlespacing*{\section}{0em}{*4}{*1}

% A break for each new section with figures printed before new section
% \newcommand{\sectionbreak}{\clearpage}
% \newcommand{\sectionbreak}{\clearpage\phantomsection} % if hyperref loaded before titlesec

% Title numbering depth
\setcounter{secnumdepth}{4}
% Table of content depth
\setcounter{tocdepth}{3}
\renewcommand\contentsname{Content} % toc title


% -----------
% MISE EN FORME TEXTES SPECIAUX
% -----------
% Additionnal Text colors
\usepackage{xcolor}
\definecolor{backcolor}{RGB}{235, 235, 235}

\newcommand{\mybox}[1]{\par\noindent\colorbox{backcolor}
{\parbox{\dimexpr\textwidth-2\fboxsep\relax}{#1}}}

\newcommand{\keyword}[1]{\textcolor{red!60!black}{#1}}
\newcommand{\advert}[1]{\textit{\textcolor{orange!80!black}{#1}}}
\newcommand{\exo}[1]{\textit{\textcolor{green!80!black}{#1}}}
%\newcommand{\codecommand}[1]{\par\noindent\colorbox{backcolor}{\texttt{#1}}}
%\newcommand{\menucommand}[1]{\par\fontfamily{pcr}\fontsize{12}{12}\selectfont\color{red}\noindent\colorbox{shadecolor}{\textit{#1}}}{\par}
\newcommand{\menucommand}[1]{\textit{\textcolor{blue!80!black}{#1}}}
\newcommand{\codecommand}[1]{\texttt{\colorbox{backcolor}{#1}}}

\newenvironment{important}{\par\color{black!80!green}\itshape}{\par}

\newsavebox{\selvestebox}
\newenvironment{redbox}
  {
   \begin{lrbox}{\selvestebox}%
   \begin{minipage}{\dimexpr\columnwidth-2\fboxsep\relax}}
  {\end{minipage}\end{lrbox}%
   \colorbox[HTML]{FF7F7F}{\usebox{\selvestebox}}
   }

% \newsavebox{\selvestebox}
\newenvironment{codebox}{
  \begin{lrbox}{\selvestebox}%
}{
  \end{lrbox}%
  \colorbox{backcolor}{\usebox{\selvestebox}}
}

% - source: http://tex.stackexchange.com/questions/82028/how-do-i-create-a-variant-of-the-snugshade-box-from-the-framed-package-to-wrap-m
\newenvironment{blueShaded}[1][D6E8F5]{
  \definecolor{shadecolor}{HTML}{#1}%
  \begin{snugshade}%
}{%
    \end{snugshade}%
}

% -- command for pandoc trick with \begin and \end -- %
\newcommand{\nopandoc}[1]{#1}

% ----------
% Pour justifier le texte apres un alignement à gauche
% Utile pour le texte en caractère tt qui ne se coupe pas
% ----------
\usepackage{ragged2e}
