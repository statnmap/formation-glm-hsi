% GIS and R Course
% author : Sébastien Rochette
% contact : sebastienrochettefr@gmail.com
% website : http://statnmap.com

% \documentclass[a4paper]{article}
% Lang EN = 1, FR = 2
% \def\Lang{\Sexpr{Lang}} % 
% -- Command to find which language is loaded in babel -- %
% http://tex.stackexchange.com/questions/287667/ifpackagewith-doesnt-behave-as-i-expected-with-global-options
\usepackage{xparse}
\ExplSyntaxOn
\NewDocumentCommand{\packageoptionsTF}{mmmm}
 {
  \stanton_package_options:nnTF { #1 } { #2 } { #3 } { #4 }
 }

\cs_new_protected:Nn \stanton_package_options:nnTF
 {
  \clist_map_inline:nn { #2 }
   {
    \clist_if_in:cnTF { opt@#1.sty } { ##1 }
     { #3 } % it's a local option
     {
      \clist_if_in:cnTF { @classoptionslist } { ##1 }
       { #3 } % it's a global option
       { #4 }
     }
   }
}
\ExplSyntaxOff

% -- Define a variable depending on language -- %
\newcommand{\Lang}{0}

\makeatletter
\@ifpackageloaded{babel}{
  \packageoptionsTF{babel}{english}{%
    \renewcommand{\Lang}{1}% english
  }{%
    \renewcommand{\Lang}{2}% french
  }
}{}
\makeatother

% -- Define specific lateX options depending on language -- %
\ifnum\Lang = 1
  % \usepackage[english]{babel}
  \usepackage{enumitem}
  \setlist{itemsep = 0pt}
  \setlist{topsep = 0pt}
\fi
\ifnum\Lang = 2
  % \usepackage[french]{babel}
\fi

% --
\usepackage[utf8]{inputenc}
\usepackage[T1]{fontenc}
\usepackage{amsmath}
%\usepackage{amssymb,amsfonts,textcomp}
\usepackage{color}
\usepackage{array}
\usepackage{hhline}
%\usepackage[linktocpage=true]{hyperref} % to have only number clickable in toc
\hypersetup{linktocpage=true, colorlinks=true, linkcolor=[RGB]{241,85,34}, citecolor=[RGB]{241,85,34}, filecolor=[RGB]{241,85,34}, urlcolor=[RGB]{241,85,34}}
%\usepackage[pdftex]{graphicx}
\usepackage{tikz}
%\usepackage{float} % To force figure to be placed where I want with H
% do not use float as it changes space between figures and their caption.
% Better use [!h] option for figures
\usepackage[normalem]{ulem} % to underline text on multiple lines

\renewcommand{\emph}[1]{\textit{#1}}

\usepackage{lmodern} % For higher definition fonts
\usepackage[font = footnotesize, labelfont = bf, margin = 1cm]{caption} %name=Fig.

% Text styles
\newcommand\textstyleInternetlink[1]{\textcolor[RGB]{21,183,214}{#1}}
\newcommand\Greytext[1]{\textcolor[RGB]{75,75,75}{#1}}
\newcommand\LightGrey[1]{\textcolor[RGB]{173,169,174}{#1}}
\newcommand\OtherGrey[1]{\textcolor[RGB]{200,200,200}{#1}}


% List styles
%\newcommand\liststyleWWviiiNumii{%
%\renewcommand\labelitemi{{\textbullet}}
%\renewcommand\labelitemii{{\textbullet}}
%\renewcommand\labelitemiii{{\textbullet}}
%\renewcommand\labelitemiv{{\textbullet}}
%}

\AtBeginDocument{
\def\labelitemi{$\bullet$}%
\def\labelitemii{$\circ$}%
\def\labelitemiii{$-$}%
\def\labelitemiv{$-$}%
}

% FIGURES
% \graphicspath{{/mnt/Data/Formation_SIG-et-R/00_Original_TD_support/img_QGIS/}{/mnt/Data/Formation_SIG-et-R/00_Original_TD_support/figureR/}{/mnt/Data/Formation_SIG-et-R/00_Original_TD_support/Figures_Pres/}{/mnt/Data/autoentrepreneur/Presentation_Produits/SRochettePresentation-img/}}

%\setlength{\abovecaptionskip}{5pt}
%\setlength{\belowcaptionskip}{10pt}


% DIMENSIONS - MARGINS
% \usepackage[top=2.4cm, bottom=2.1cm, outer=2cm, inner=4cm, headheight=40pt]{geometry} %heightrounded
%\setlength\hoffset{0cm}
%\setlength\voffset{0cm}
\setlength\topmargin{-2cm}
%\setlength\headheight{2cm}
\setlength\headsep{0.50cm}
%\setlength\textheight{25.7cm}
\setlength\footskip{1.1cm}
\setlength{\parindent}{0em}
%\setlength{\parskip}{0em}
\setlength\belowcaptionskip{5pt}
\setlength\abovecaptionskip{8pt}
%\let\oldfigure\figure
%\let\oldtable\table
%\def\figure{\setlength\abovecaptionskip{5pt}\oldfigure}
%\def\table{\setlength\belowcaptionskip{1cm}\oldtable}
%\usepackage{caption} %[font = small]
%\setlength{\abovecaptionskip}{10pt plus 5pt minus 5pt}
%\captionsetup[table]{skip = 10pt}
%\captionsetup[figure]{skip = 10pt}
%\setlength\longindentation 0.60\textwidth

% \usepackage{lastpage} % to calculate number of pages to put in footer
\usepackage{pageslts}

% ----------
% BACKGROUND
% ----------
\usepackage{eso-pic}
\newcommand\BackgroundPic{%
\put(0,0){%
\parbox[b][\paperheight]{\paperwidth}{%
\vfill
\centering
  \includegraphics[width=\paperwidth,height=\paperheight]{Background_light_topdown_ThinkR.png}%
\vfill
}}}
\newcommand\BackgroundPicTitle{%
\put(0,0){%
\parbox[b][\paperheight]{\paperwidth}{%
\vfill
\centering
  \includegraphics[width=\paperwidth,height=\paperheight]{Background_Title_light_ThinkR.png}%
\vfill
}}}

%...and this immediately after \begin{document}:
%\AddToShipoutPicture*{\BackgroundPic}
%The * will make sure that the background picture will only be put on one page.
%If you wish to use the picture on multiple pages, skip the *:
%\AddToShipoutPicture{\BackgroundPic}
%Then use this command to stop using the background picture:
%\ClearShipoutPicture


% ----------
% MISE EN FORME DES TITRES
% ----------
%\renewcommand{\thesection}{\arabic{section}}
%\renewcommand{\thesubsection}{\arabic{section}.\arabic{subsection}}
%\renewcommand{\thesubsubsection}{\arabic{section}.\arabic{subsection}.\arabic{subsubsection}}

%% Provide a definition to \subparagraph to keep titlesec happy
\let\subparagraph\oldsubparagraph
\let\paragraph\oldparagraph
%% Load titlesec
%\usepackage[compact]{titlesec}
\usepackage{titlesec}
%% Revert \subparagraph to the Rmd definition
% Redefines (sub)paragraphs to behave more like sections
\ifx\paragraph\undefined\else
\let\oldparagraph\paragraph
\renewcommand{\paragraph}[1]{\oldparagraph{#1}\mbox{}}
\fi
\ifx\subparagraph\undefined\else
\let\oldsubparagraph\subparagraph
\renewcommand{\subparagraph}[1]{\oldsubparagraph{#1}\mbox{}}
\fi


%\titleformat{(command)}[(shape)]{(format)}{(label)}{(sep)}{(before)}[(after)]
\titleformat{\section}%
[block]% style du titre (block, hang, display, runin, leftmargin, drop, wrap)
{\Large\color[RGB]{21,183,214}}%changement de fonte commun au numéro et au titre
{\thesection.}% spécification du numéro
{1em}% 1em espace entre le numéro et le titre
{}% changement de fonte du titre

\titleformat{\subsection}%
[block]% style du titre (block, hang, display, runin, leftmargin, drop, wrap)
{\large\bfseries} %\itshape {\Large\bfseries\color[RGB]{30,115,190}}%changement de fonte commun au numéro et au titre
{\thesubsection.}% spécification du numéro
{1em}% {1em}% espace entre le numéro et le titre
{}% changement de fonte du titre

\titleformat{\subsubsection}%
[block]% style du titre (block, hang, display, runin, leftmargin, drop, wrap)
{\large} %\itshape {\Large\bfseries\color[RGB]{30,115,190}}%changement de fonte commun au numéro et au titre
{\thesubsubsection.}% spécification du numéro
{1em}% {1em}% espace entre le numéro et le titre
{}% changement de fonte du titre

\titleformat{\paragraph}%
[block]% style du titre (block, hang, display, runin, leftmargin, drop, wrap)
{\itshape} %\itshape {\Large\bfseries\color[RGB]{30,115,190}}%changement de fonte commun au numéro et au titre
{\theparagraph.}% spécification du numéro
{1em}% {1em}% espace entre le numéro et le titre
{}% changement de fonte du titre

\titleformat{\subparagraph}%
[block]% style du titre (block, hang, display, runin, leftmargin, drop, wrap)
{\itshape} %\itshape {\Large\bfseries\color[RGB]{30,115,190}}%changement de fonte commun au numéro et au titre
{\thesubparagraph.}% spécification du numéro
{1em}% {1em}% espace entre le numéro et le titre
{}% changement de fonte du titre


%\titlespacing*{(command)}{(left)}{(beforesep)}{(aftersep)}[(right)]
\titlespacing*{\section}{0em}{*4}{*1}

% A break for each new section with figures printed before new section
% \newcommand{\sectionbreak}{\clearpage}
% \newcommand{\sectionbreak}{\clearpage\phantomsection} % if hyperref loaded before titlesec

% Title numbering depth
\setcounter{secnumdepth}{4}
% Table of content depth
\setcounter{tocdepth}{3}
\renewcommand\contentsname{Content} % toc title


% -----------
% MISE EN FORME TEXTES SPECIAUX
% -----------
% Additionnal Text colors
\usepackage{xcolor}
\definecolor{backcolor}{RGB}{235, 235, 235}

\newcommand{\mybox}[1]{\par\noindent\colorbox{backcolor}
{\parbox{\dimexpr\textwidth-2\fboxsep\relax}{#1}}}

\newcommand{\keyword}[1]{\textcolor{red!60!black}{#1}}
\newcommand{\advert}[1]{\textit{\textcolor{orange!80!black}{#1}}}
\newcommand{\exo}[1]{\textit{\textcolor{green!80!black}{#1}}}
%\newcommand{\codecommand}[1]{\par\noindent\colorbox{backcolor}{\texttt{#1}}}
%\newcommand{\menucommand}[1]{\par\fontfamily{pcr}\fontsize{12}{12}\selectfont\color{red}\noindent\colorbox{shadecolor}{\textit{#1}}}{\par}
\newcommand{\menucommand}[1]{\textit{\textcolor{blue!80!black}{#1}}}
\newcommand{\codecommand}[1]{\texttt{\colorbox{backcolor}{#1}}}

\newenvironment{important}{\par\color{black!80!green}\itshape}{\par}

\newsavebox{\selvestebox}
\newenvironment{redbox}
  {
   \begin{lrbox}{\selvestebox}%
   \begin{minipage}{\dimexpr\columnwidth-2\fboxsep\relax}}
  {\end{minipage}\end{lrbox}%
   \colorbox[HTML]{FF7F7F}{\usebox{\selvestebox}}
   }

% \newsavebox{\selvestebox}
\newenvironment{codebox}{
  \begin{lrbox}{\selvestebox}%
}{
  \end{lrbox}%
  \colorbox{backcolor}{\usebox{\selvestebox}}
}

% - source: http://tex.stackexchange.com/questions/82028/how-do-i-create-a-variant-of-the-snugshade-box-from-the-framed-package-to-wrap-m
\newenvironment{blueShaded}[1][D6E8F5]{
  \definecolor{shadecolor}{HTML}{#1}%
  \begin{snugshade}%
}{%
    \end{snugshade}%
}

% -- command for pandoc trick with \begin and \end -- %
\newcommand{\nopandoc}[1]{#1}

% ----------
% Pour justifier le texte apres un alignement à gauche
% Utile pour le texte en caractère tt qui ne se coupe pas
% ----------
\usepackage{ragged2e}

% \input{/mnt/Data/ThinkR/Gitlab/thinkridentity/inst/templates/latex/MiseEnFormeTitreFormationRmd_NoSectionBreak.tex}

\ifnum\Lang = 1
% ---------------
% HEADER / FOOTER
% ---------------
\usepackage{fancyhdr}
\pagestyle{fancy}
%\fancyhf{}
\fancyhead[L]{
\hspace{-1cm}\LARGE{\textbf{\href{https://thinkr.fr/}{ThinkR}}}\\
%\hspace{-1cm}\normalsize{\Greytext{Modélisation statistique, analyse de données géographiques et cartographie}}\\
\hspace{-1cm}\normalsize{\Greytext{Courses and consulting for R}}\\
%\normalsize{\href{url:http://statnmap.com/}{\textstyleInternetlink{http://statnmap.com/}}}
}
\fancyhead[R]{\Greytext{\textit{Created on \today}\hspace{-1cm}}}
%
\fancyfoot[L]{\Greytext{\hspace{-1cm}Sébastien Rochette}}
\fancyfoot[C]{\href{mailto:sebastien@thinkr.fr}{sebastien@thinkr.fr}}
\fancyfoot[R]{\Greytext{\thepage\ / \pageref*{LastPage}\hspace{-1cm}}}
%\cfoot{Page \thepage\ (\theCurrentPage) of \lastpageref{LastPages}}
\renewcommand{\headrulewidth}{0pt}
\renewcommand{\footrulewidth}{0pt}
%\fancyheadoffset{length}


\fi
\ifnum\Lang = 2
% ---------------
% HEADER / FOOTER
% ---------------
\usepackage{fancyhdr}
\pagestyle{fancy}
%\fancyhf{}
\fancyhead[L]{
\hspace{-1cm}\LARGE{\textbf{\href{https://thinkr.fr/}{ThinkR}}}\\
%\hspace{-1cm}\normalsize{\Greytext{Modélisation statistique, analyse de données géographiques et cartographie}}\\
\hspace{-1cm}\normalsize{\Greytext{Formation et consultance sur R}}\\
%\normalsize{\href{url:http://sebrock.fr/}{\textstyleInternetlink{http://sebrock.fr/}}}
}
\fancyhead[R]{\Greytext{\textit{Créé le \today}\hspace{-1cm}}}
%
\fancyfoot[L]{\Greytext{\hspace{-1cm}Sébastien Rochette}}
\fancyfoot[C]{\href{mailto:sebastien@thinkr.fr}{sebastien@thinkr.fr}}
% \fancyfoot[R]{\Greytext{\thepage\ / \pageref*{LastPage}\hspace{-1cm}}}
%\cfoot{Page \thepage\ (\theCurrentPage) of \lastpageref{LastPages}}
\fancyfoot[R]{\Greytext{\thepage\ / \pageref*{LastPage}\hspace{-1cm}}}
\renewcommand{\headrulewidth}{0pt}
\renewcommand{\footrulewidth}{0pt}
%\fancyheadoffset{length}

\fi

% -- Graphic path -- %
% \graphicspath{{/mnt/Data/Formation_SIG-et-R/00_Original_TD_support/img_QGIS/}{/mnt/Data/Formation_SIG-et-R/00_Original_TD_support/figureR/}{/mnt/Data/Formation_SIG-et-R/00_Original_TD_support/Figures_Pres/}{/mnt/Data/autoentrepreneur/Presentation_Produits/SRochettePresentation-img/}}
\graphicspath{{/mnt/Data/ThinkR/Gitlab/thinkridentity/inst/img/}}

% \title{GIS and R Course}
% \author{Sébastien Rochette}

\setcounter{section}{0} % Value for first section


\hypersetup{pdfauthor=Sébastien Rochette, pdftitle=Formation ThinkR, pdfsubject=Formation à R, pdfkeywords=R, pdfcreator=pdflatex}

%----------------------------------------------------------------------------------------
% TITLE PAGE
%----------------------------------------------------------------------------------------

\newcommand*{\titleGM}{\begingroup % Create the command for including the title page in the document
\hbox{ % Horizontal box
%\hspace*{0.2\textwidth} % Whitespace to the left of the title page
%\hspace*{0.2\textwidth}
\OtherGrey{\rule{1pt}{\textheight}} % Vertical line
\hspace*{0.05\textwidth} % Whitespace between the vertical line and title page text
\parbox[b]{0.75\textwidth}{ % Paragraph box which restricts text to less than the width of the page

{\noindent\Huge\bfseries\textstyleInternetlink{ Formation à R}}\\[2\baselineskip] % Title
{\large{Modélisation avec les GLM}}\\[4\baselineskip] % Tagline or further description
{\Large \textsc{Sébastien Rochette, ThinkR}} % Author name

\vspace{0.5\textheight} % Whitespace between the title block and the publisher
{\noindent \href{url:https://thinkr.fr}{ThinkR}}\\[\baselineskip] % Publisher and logo
}}
\endgroup}

\AtBeginDocument{\let\maketitle\relax}

